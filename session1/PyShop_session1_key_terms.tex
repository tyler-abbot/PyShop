\documentclass{article}
\usepackage[utf8]{inputenc}

\title{Key Terms, session 1}
\author{Tyler Abbot}
\date{September 2015}

\begin{document}

\maketitle
These are the key terms from session 1.  If you think of any that should be added to this list, let me know!

\begin{itemize}
\item Object oriented programming - An object oriented programming language is built around objects that incorporate both information and functioality.  Other languages, such as Java or C++, are also object oriented or multi-paradigm, but Pythons 'method' attributes make object oriented programming quite easy to learn and use.

\item Integrated Development Environment (IDE) - This is a piece of software that incorporates text editor, debugger, and compiler into one.

\item GitHub - A repository hosting service based on the Git distributed revision control system.  

\item Script - A file containing a program.  For instance if you write a hello world program and save it as a .py file, that's a script.

\item Docstring - a docstring is a 'string literal' that informs the reader(not the computer!)  of the usage for a program, function, mod-ule,  etc. 

\item Import statements - These come at the beginning of a Python program and allow you to run pre-written python code, which "imports" funcitonality into your program.

\item Variable scope - Scope refers to where a variable is defined.  A variable with local scope is only available to the function currently being run, while a global variable is available to the entire program.  This idea is related to the "namespace", or the set of variables currently defined.  There is a local and global namespace, each of which contain the local and global variables, respectively.

\item Module - A set of function definitions, class definitions, and executables which are grouped together under functionality.  You can import these modules at the beginning of a Python program.

\item Package - A set of modules that all share some functionality.

\item Array - Like a matrix, but with no limit in dimensions.

\item Web Scraping - Searching through web pages for information.  For instance, if you need to get a form variable to log into a website, you would scrape the page for its value.

\end{itemize}
\end{document}